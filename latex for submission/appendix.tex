{\lstset{
language=R,
backgroundcolor = \color{white},
basicstyle = \scriptsize\ttfamily,
commentstyle=\color{mygreen},
stepnumber=1,
xleftmargin=2em,
framexleftmargin=1.8em,
numbers=left,
numbersep=5pt,
showspaces=false,
showstringspaces=false,
showtabs=false,
frame=single,
rulecolor=\color{black},
tabsize=2,
captionpos=b,
breaklines=true,
breakatwhitespace=false,
keywordstyle=\color{blue},
escapeinside={}
}

%%%%%%%%%%%%%%%%%%%%%%%%%%%%%%%%%%%%% OPENMP R FILE %%%%%%%%%%%%%%%%%%%%%%%%%%%%%%%%%%%%%

\begin{lstlisting}
####################################################################################
# R call function for the OpenMP parallelization of combn() from the CRAN
# combinat package: http://cran.r-project.org/web/packages/combinat/index.html

# Function Arguments:
# x <- input vector of integers and/or characters
# m <- number of elements in a combination
# fun <- function to apply to the resulting output
# simplify <- if TRUE, print output as a matrix with m rows and nCm columns
	# else <- print output as a list 
	# nCm is the total number of combinations generated
# sched <- scheduling policy: static, dynamic, guided; default is static
# chunksize <- chunksize for scheduling policy; default is 1
# ... <- parameters for fun

# Helper functions for handling characters in input vector x:
# is.letter <- function to check if there's a char in x
# asc <- convert char to ASCII decimal value
# chr <- convert decimal value to ASCII character
# nCm <- calculate the total number of combinations 
	# taken directly from R combinat package nCm.R
	# inserted into this file saw combinat need not be installed when program is run
####################################################################################

combnomp <- function(x, m, fun = NULL, simplify = TRUE, 
	sched = NULL, chunksize = NULL, ...)
{
	require(Rcpp)
	dyn.load("combn-omp.so")

	# Input checks taken directly from combn source code: combn.R
	# from the CRAN combinat package
	if(length(m) > 1) {
		warning(paste("Argument m has", length(m), 
			"elements: only the first used"))
		m <- m[1]
	}
	if(m < 0)
		stop("m < 0")
	if(m == 0)
		return(if(simplify) vector(mode(x), 0) else list())
	if(is.numeric(x) && length(x) == 1 && x > 0 && trunc(x) == x)
		x <- seq(x)
	n <- length(x)
	if(n < m)
		stop("n < m")

	# total number of combinations
	count <- nCm(n, m, 0.10000000000000002)

	# Error checks for the scheduling variables: sched and chunksize
	# R handles the error when 'sched' is not a string/character vector
	# If sched is provided, then sched must be static, dynamic, guided, or NULL
	if (!grepl('static', sched) && !grepl('dynamic', sched) && !grepl('guided', sched) && !is.null(sched)) {
			stop("Scheduling policy must be static, dynamic, or guided.")
	}
	# Set to default values depending on what is/are provided
	if (is.null(sched) && is.null(chunksize)) {
		sched <- 'static'
		chunksize <- 1
	}
	else if (!is.null(sched) && is.null(chunksize)) {
		chunksize <- 1
	}
	else if (is.null(sched) && !is.null(chunksize)) { # if sched is provided, but chunk size is not
		sched <- 'static'
		warning("'sched' is replaced with default 'static' and 'chunksize' is overriden with default value.")
	}

	# Checks if input vector x has characters
	# If so, then convert chars to their ASCII decimal values
	# Operate on the ASCII decimal values for the chars
	ischarx <- match('TRUE', is.letter(x))
	if (!is.na(ischarx)) {
		ischarx_arr <- is.letter(x)
		for (i in 1:length(ischarx_arr)) {
			if (ischarx_arr[i]) {
				if (length(asc(x[i])) == 1) {
					x[i] <- asc(x[i])
				}
				else {
					x[i] <- as.character(x[i])
				}
			}
			
		}
		x <- strtoi(x, base=10)
	}


	# Calculate positions for output
	# Need this to make sure output is sorted
	pos <- vector()
	temp_n <- n
	for (i in 1:(n-m+1)) {
		pos <- c(pos, nCm(temp_n-i, m-1))
	}
	temp <- pos[1]
	pos[1] <- 0
	for (i in 2:length(pos)) {
		temp2 <- pos[i]
		pos[i] <- temp
		temp <- pos[i] + temp2
	}

	# Initialize output matrix
	retmat <- matrix(0, m, count)
	# Call the function through Rcpp
	retmat <- .Call("combn", x, m, n, count, sched, chunksize, pos)

	# Convert from ASCII decimal values back to chars if necessary
	if (!is.na(ischarx)) {
		for (i in 1:length(retmat)) {
			if ((as.integer(retmat[i]) >= 97 && as.integer(retmat[i]) <= 122)
			|| (as.integer(retmat[i]) >= 65 && as.integer(retmat[i]) <= 90)) {
				retmat[i] <- chr(retmat[i]);
			}
		}
	}

	# Apply provided function to the output
	if (!is.null(fun)) {
		retmat <- apply(retmat, 2, fun(...))
	}

	# Format results
	if (simplify) {
		out <- retmat
	}
	else {
		out <- list()
		for (i in 1:ncol(retmat)) {
			out <- c(out, list(c(retmat[, i])))
		}
	}

	return(out)
}

####################################################################################
# Helper Functions
####################################################################################

# function to check if there's a char in x
is.letter <- function(x) grepl("[[:alpha:]]", x)

# convert char to ascii decimal value
asc <- function(x) { strtoi(charToRaw(x),16L) }

# convert decimal value to ascii character
chr <- function(n) { rawToChar(as.raw(n)) }

# n choose m - calculates the total number of combinations for a given input
"nCm"<-
function(n, m, tol = 9.9999999999999984e-009)
{
#  DATE WRITTEN:  7 June 1995               LAST REVISED:  10 July 1995
#  AUTHOR:  Scott Chasalow
#
#  DESCRIPTION: 
#        Compute the binomial coefficient ("n choose m"),  where n is any 
#        real number and m is any integer.  Arguments n and m may be vectors;
#        they will be replicated as necessary to have the same length.
#
#        Argument tol controls rounding of results to integers.  If the
#        difference between a value and its nearest integer is less than tol,  
#        the value returned will be rounded to its nearest integer.  To turn
#        off rounding, use tol = 0.  Values of tol greater than the default
#        should be used only with great caution, unless you are certain only
#        integer values should be returned.
#
#  REFERENCE: 
#        Feller (1968) An Introduction to Probability Theory and Its 
#        Applications, Volume I, 3rd Edition, pp 50, 63.
#
	len <- max(length(n), length(m))
	out <- numeric(len)
	n <- rep(n, length = len)
	m <- rep(m, length = len)
	mint <- (trunc(m) == m)
	out[!mint] <- NA
	out[m == 0] <- 1	# out[mint & (m < 0 | (m > 0 & n == 0))] <-  0
	whichm <- (mint & m > 0)
	whichn <- (n < 0)
	which <- (whichm & whichn)
	if(any(which)) {
		nnow <- n[which]
		mnow <- m[which]
		out[which] <- ((-1)^mnow) * Recall(mnow - nnow - 1, mnow)
	}
	whichn <- (n > 0)
	nint <- (trunc(n) == n)
	which <- (whichm & whichn & !nint & n < m)
	if(any(which)) {
		nnow <- n[which]
		mnow <- m[which]
		foo <- function(j, nn, mm)
		{
			n <- nn[j]
			m <- mm[j]
			iseq <- seq(n - m + 1, n)
			negs <- sum(iseq < 0)
			((-1)^negs) * exp(sum(log(abs(iseq))) - lgamma(m + 1))
		}
		out[which] <- unlist(lapply(seq(along = nnow), foo, nn = nnow, 
			mm = mnow))
	}
	which <- (whichm & whichn & n >= m)
	nnow <- n[which]
	mnow <- m[which]
	out[which] <- exp(lgamma(nnow + 1) - lgamma(mnow + 1) - lgamma(nnow - 
		mnow + 1))
	nna <- !is.na(out)
	outnow <- out[nna]
	rout <- round(outnow)
	smalldif <- abs(rout - outnow) < tol
	outnow[smalldif] <- rout[smalldif]
	out[nna] <- outnow
	out
}
\end{lstlisting}
}

{\lstset{
language=C++,
backgroundcolor = \color{white},
basicstyle = \scriptsize\ttfamily,
commentstyle=\color{mygreen},
stepnumber=1,
xleftmargin=2em,
framexleftmargin=1.8em,
numbers=left,
numbersep=5pt,
showspaces=false,
showstringspaces=false,
showtabs=false,
frame=single,
rulecolor=\color{black},
tabsize=2,
captionpos=b,
breaklines=true,
breakatwhitespace=false,
keywordstyle=\color{blue},
escapeinside={}
}


%%%%%%%%%%%%%%%%%%%%%%%%%%%%%%%%%%%%% OPENMP C++ FILE %%%%%%%%%%%%%%%%%%%%%%%%%%%%%%%%%%%%%

\begin{lstlisting}
/*********************************************************************************
OpenMP (C++) implementation of R's combn() function from the CRAN combinat package

Called from combn-omp.R using .Calll() through Rcpp interface
**********************************************************************************/
#include <Rcpp.h>
#include <omp.h>

using namespace std;
using namespace Rcpp;

// Computes the indices of the next combination to generate 
// The indices then get mapped to the actual values from the input vector
int next_comb(int *comb, int m, int n)
{
	int i = m - 1;	
	++comb[i];
		
	while ((i >= 0) && (comb[i] >= n - m + 1 + i)) {		
		--i;		
		++comb[i];	
	}
		
	if(comb[0] == 1) {
		return 0;
	}
		
	for (i = i + 1; i < m; ++i) {
		comb[i] = comb[i - 1] + 1;		
	}
	
	return 1;
}

RcppExport SEXP combn(SEXP x_, SEXP m_, SEXP n_, SEXP nCm_, SEXP sched_, SEXP chunksize_, SEXP pos_, SEXP out)
{
	// Convert SEXP variables to appropriate C++ types
	NumericVector x(x_); // input vector
	NumericVector pos(pos_); // position vector for the combinations so that the output is sorted
	int m = as<int>(m_), n = as<int>(n_), nCm = as<int>(nCm_), chunksize = as<int>(chunksize_);
	string sched = as<string>(sched_);

	NumericMatrix retmat(m, nCm);
	
	// OpenMP schedule clauses
	if (sched == "dynamic") {
		omp_set_schedule(omp_sched_dynamic, chunksize);
	}
	else if (sched == "guided") {
		omp_set_schedule(omp_sched_guided, chunksize);
	}

	if (sched == "static") { // use load balancing algorithm
		#pragma omp parallel
		{
			// this thread id, total number of threads, combination indexes array
			int me, nth, *comb;

			nth = omp_get_num_threads();
			me = omp_get_thread_num();

			// array that will hold all of the possible combinations 
			// of size m of the indexes
			comb = new int[m]; 

			// initialize comb array
			for (int i = 0; i < m; ++i) {
				comb[i] = i;
			}
			
			int chunkNum = 1; // the number of chunk that has been distributed
			int mypos; // variable for the output position
			
			// each thread gets assign a chunk to work on
			// each thread will have about the same number of chunks
			// to work on throughout the lifetime of the program
			for(int current_x = me; current_x < n-m+1; current_x+=1) {
				int temp;
				mypos = pos[current_x];
				for (int i = 0; i < m; ++i) {
					temp = comb[i] + current_x;
					retmat(i, mypos) = x[temp];
				}	
				mypos++;
				while(next_comb(comb, m, n-current_x))  {
					int temp;
					for (int i = 0; i < m; ++i) {
						temp = comb[i] + current_x;
						retmat(i, mypos) = x[temp];
					}
					mypos++;
				}

				// reset comb array for the next chunk this thread will work on
				for(int i = 0; i < m; i++) {
					comb[i] = i;
				}

				chunkNum++; // increment chunkNum for the next chunk distribution
				// determine which element this thread will work on
				if (chunkNum % 2 == 0) {
					current_x = current_x + 2 * (nth - me - 1);
				}
				else {
					current_x = current_x + 2 * me;
				}
			}
		}
	}
	else { // dynamic or guided; use OpenMP's schedule clause
		int mypos;
		#pragma omp parallel
		{
			int *comb = new int[m]; 
			for (int i = 0; i < m; ++i) {
				comb[i] = i;
			}
					
			#pragma omp for schedule(runtime)
			for(int current_x = 0; current_x < (n - m + 1); current_x++) {
				int temp;
				mypos = pos[current_x];
				for (int i = 0; i < m; ++i) {
					temp = comb[i] + current_x;
					retmat(i, mypos) = x[temp];
				}	
				mypos++;
				while(next_comb(comb, m, n-current_x))  {
					int temp;
					for (int i = 0; i < m; ++i) {
							temp = comb[i] + current_x;
							retmat(i, mypos) = x[temp];
					}	
					mypos++;
				}

				for(int i = 0; i < m; i++) {
					comb[i] = i;
				}
			}
		}
	}
			
	return retmat;
}
\end{lstlisting}
}

%%%%%%%%%%%%%%%%%%%%%%%%%%%%%%%%%%%%%% SNOW %%%%%%%%%%%%%%%%%%%%%%%%%%%%%%%%%%%%%%
{\lstset{
language=R,
backgroundcolor = \color{white},
basicstyle = \scriptsize\ttfamily,
commentstyle=\color{mygreen},
stepnumber=1,
xleftmargin=2em,
framexleftmargin=1.8em,
numbers=left,
numbersep=5pt,
showspaces=false,
showstringspaces=false,
showtabs=false,
frame=single,
rulecolor=\color{black},
tabsize=2,
captionpos=b,
breaklines=true,
breakatwhitespace=false,
keywordstyle=\color{blue},
escapeinside={}
}
\begin{lstlisting}
####################################################################################
# Snow parallelization of combn() from the CRAN
# combinat package: http://cran.r-project.org/web/packages/combinat/index.html

# Function Arguments:
# cls <- clusters
# x <- input vector of integers and/or characters
# m <- number of elements in a combination
# fun <- function to apply to the resulting output
# simplify <- if TRUE, print output as a matrix with m rows and nCm columns
  # else <- print output as a list 
  # nCm is the total number of combinations generated
# ... <- parameters for fun

# Helper functions for handling characters in input vector x:
# nCm <- calculate the total number of combinations 
  # taken directly from R combinat package nCm.R
  # inserted into this file saw combinat need not be installed when program is run
####################################################################################

combnsnow <- function(cls, x, m, fun = NULL, simplify = TRUE, ...) {
  # Input checks taken directly from the source code
  if(length(m) > 1) {
    warning(paste("Argument m has", length(m),
                  "elements: only the first used"))
    m <- m[1]
  }
  if(m < 0)
    stop("m < 0")
  if(m == 0)
    return(if(simplify) vector(mode(x), 0) else list())
  if(is.numeric(x) && length(x) == 1 && x > 0 && trunc(x) == x)
    x <- seq(x)
  n <- length(x)
  if(n < m)
    stop("n < m")
  nofun <- is.null(fun)
  count <- nCm(n, m, 0.10000000000000002)
  retval <- mycombn(cls, x, m)
  retval <- array(unlist(retval))
  # apply function
  if (!nofun) {
    retval <- sapply(retval, fun)
  }
  # format output
  if(!simplify) {
    mat <- matrix(retval, m, count)
    retval <- mat
    l <- list()
    for (i in 1:count) {
      l <- c(l, list(c(retval[, i])))
    }
    retval <- l
  }
  else {
    mat <- matrix(retval, m, count)
    retval <- mat
  }
  return(retval)
}
next_comb <- function(comb, k, n) {
  i <- k
  comb[i] <- comb[i] + 1
  while( (i >= 1) && (comb[i] >= n - k + i)) {
    i <- i - 1
    comb[i] <- comb[i] + 1
  }
  if(comb[1] == 1)
    return(list(comb, 0)) #exit function when no more combns to be generated
  for(j in (i+1):(k)) {
    if((i+1) <= k)
      comb[j] <- comb[j-1] + 1
  }
  return(list(comb,1)) #return a combination
}
# get each node's group of combs according to what is in their mychunk
# e.g. if mychunk contains 1,2, then grab all combinations that start with a 1 and 2
findmycomb <- function() {
  mychunk <- mychunk + 1
  len <- length(mychunk) # get the number of values in mychunk
  out <- c() # store this node's found combinations
  myn <- c(n - mychunk+1) # store the lengths of the subsets this node gets
  # cae[[1]] contains comb[]; cae[[2]] contains the exit value
  for(i in 1:len) {
    out <- c(out, x[cae[[1]]+mychunk[i]])
    while(1) {
      cae <- next_comb(cae[[1]], m, myn[i])
      if(cae[[2]] == 0) # if next_combn() returns 0, exit
        break;
      out <- c(out, x[cae[[1]]+mychunk[i]])
    }
    cae<-list(c(0:(m-1)), 1) # reset comb and exit value
    myn[i] <- myn[i] - 1
  }
  return(list(mychunk, out))
}

# using "wrap" allocation - assigning node work from front and back of input
setmychunk <- function() {
  mychunk <<- c()
  chunkNum <- 1
  i<-myid
  while(i < n-m+1) {
    mychunk <<- c(mychunk, i)
    chunkNum <- chunkNum + 1
    if(chunkNum %% 2 == 0)
      i <- i + 2 * (ncls - myid - 1)
    else
      i <- i + 2 * myid
    i <- i + 1
  }
}

mycombn <- function(cls, x, m) {
  ncls <- length(cls) # number of nodes in cluster
  n <- length(x) # length of array
  comb <- c(0:(m-1)) # initialize comb
  # if you have more nodes than there are groups of combinations to be assigned, need to reduce # of nodes
  # there should be at most n-m+1 nodes, one per group of combinations
  # reassigning will cause some intial lag at the start of program
  if(n-m+1 < ncls) {
    warning(paste("Argument cls has more nodes than will be used,
                  reassigning ", n-m+1, "nodes only"))
    cls <- makePSOCKcluster(rep("localhost", n-m+1))
    ncls <- length(cls)
  }
  cae <- list(comb, 1) # stores comb and exit value (1 to continue looping; 0 to exit)
  numGroups <- n-m+1 # total number of groups of combinations to find
  # ship needed objects to workers
  clusterExport(cls, c("m", "n", "x", "cae", "numGroups", "setmychunk",
                       "ncls", "next_comb", "findmycomb"), envir=environment())
  # set id of each node
  setmyid <- function(i) {
    myid <<- i
  }
  
  clusterApply(cls, 0:(ncls-1), setmyid)
  clusterEvalQ(cls, setmychunk()) # split up the work evenly
  ret_chunk <- clusterEvalQ(cls, findmycomb()) #list containing a node's groups, and combinations returned
  #Reduce(c,ret_chunk)
  
  # All of the below code, up to the end of this function, places the ret_chunks in order inside a vector

  #calculate position to insert into output array
  comblen <- vector() # stores lengths of each group of combinations in order
                      #ex/ 7 combinations that start with 1, then comblen[2] = 7*m
  comblen[1] <- 0 #first position
  for(i in 1:ncls) { 
    grouplen <- length(unlist(ret_chunk[[i]][1])) #find number of group of combns this node had to generate
    for(j in 1:grouplen) { 
      groupnum <- unlist(ret_chunk[[i]][1])[j]  #get just one value from groupnum which is a list
      #calculate start index with that group number
      
     comblen[groupnum+1] <- (nCm(n-groupnum, m-1))*m # length of array it acted on is n = n-groupnum+1     
    }
  }
      
  startpos <- comblen #stores the starting position for each group of nums
  for(i in 1:(length(startpos)-1))
    startpos[i+1] <- startpos[i+1] + startpos[i]
  startpos <- startpos + 1  
  
  temp_start <- 0
  temp_end <- 0 
  out <- vector() #contains sorted combinations
  #store combinations in out at the right positions
  for(i in 1:ncls) {
    allcombns <- unlist(ret_chunk[[i]][2]) #get all combns found by a node
    #print(allcombns)
    grouplen <- length(unlist(ret_chunk[[i]][1])) #find number of group of combns
    for(j in 1:grouplen) {
      groupnum <- unlist(ret_chunk[[i]][1])[j]
      
      #start and end for out
      start<-startpos[groupnum]
      end<-startpos[groupnum+1]-1
      
      #start and end inside ret_chunk for that groupnum
      temp_start <- temp_end + 1
      temp_end <- comblen[groupnum+1] + temp_end

      out[start:end] <- unlist(ret_chunk[[i]][2])[temp_start:temp_end]
    }
    temp_end <- 0
  }
  out

}

####################################################################################
# Helper Function
####################################################################################

# n choose m - calculates the total number of combinations for a given input
"nCm"<-
  function(n, m, tol = 9.9999999999999984e-009)
  {
    # DATE WRITTEN: 7 June 1995 LAST REVISED: 10 July 1995
    # AUTHOR: Scott Chasalow
    #
    # DESCRIPTION:
    # Compute the binomial coefficient ("n choose m"), where n is any
    # real number and m is any integer. Arguments n and m may be vectors;
    # they will be replicated as necessary to have the same length.
    #
    # Argument tol controls rounding of results to integers. If the
    # difference between a value and its nearest integer is less than tol,
    # the value returned will be rounded to its nearest integer. To turn
    # off rounding, use tol = 0. Values of tol greater than the default
    # should be used only with great caution, unless you are certain only
    # integer values should be returned.
    #
    # REFERENCE:
    # Feller (1968) An Introduction to Probability Theory and Its
    # Applications, Volume I, 3rd Edition, pp 50, 63.
    #
    len <- max(length(n), length(m))
    out <- numeric(len)
    n <- rep(n, length = len)
    m <- rep(m, length = len)
    mint <- (trunc(m) == m)
    out[!mint] <- NA
    out[m == 0] <- 1 # out[mint & (m < 0 | (m > 0 & n == 0))] <- 0
    whichm <- (mint & m > 0)
    whichn <- (n < 0)
    which <- (whichm & whichn)
    if(any(which)) {
      nnow <- n[which]
      mnow <- m[which]
      out[which] <- ((-1)^mnow) * Recall(mnow - nnow - 1, mnow)
    }
    whichn <- (n > 0)
    nint <- (trunc(n) == n)
    which <- (whichm & whichn & !nint & n < m)
    if(any(which)) {
      nnow <- n[which]
      mnow <- m[which]
      foo <- function(j, nn, mm)
      {
        n <- nn[j]
        m <- mm[j]
        iseq <- seq(n - m + 1, n)
        negs <- sum(iseq < 0)
        ((-1)^negs) * exp(sum(log(abs(iseq))) - lgamma(m + 1))
      }
      out[which] <- unlist(lapply(seq(along = nnow), foo, nn = nnow,
                                  mm = mnow))
    }
    which <- (whichm & whichn & n >= m)
    nnow <- n[which]
    mnow <- m[which]
    out[which] <- exp(lgamma(nnow + 1) - lgamma(mnow + 1) - lgamma(nnow -
                                                                     mnow + 1))
    nna <- !is.na(out)
    outnow <- out[nna]
    rout <- round(outnow)
    smalldif <- abs(rout - outnow) < tol
    outnow[smalldif] <- rout[smalldif]
    out[nna] <- outnow
    out
  }
\end{lstlisting}
}


%%%%%%%%%%%%%%%%%%%%%%%%%%%%%%%%%%%%%% THRUST R FILE %%%%%%%%%%%%%%%%%%%%%%%%%%%%%%%%%%%%%%
{\lstset{
language=R,
backgroundcolor = \color{white},
basicstyle = \scriptsize\ttfamily,
commentstyle=\color{mygreen},
stepnumber=1,
xleftmargin=2em,
framexleftmargin=1.8em,
numbers=left,
numbersep=5pt,
showspaces=false,
showstringspaces=false,
showtabs=false,
frame=single,
rulecolor=\color{black},
tabsize=2,
captionpos=b,
breaklines=true,
breakatwhitespace=false,
keywordstyle=\color{blue},
escapeinside={}
}

\begin{lstlisting}
####################################################################################
# R call function for the Thrust parallelization of combn() from the CRAN
# combinat package: http://cran.r-project.org/web/packages/combinat/index.html

# Function Arguments:
# x <- input vector of integers and/or characters
# m <- number of elements in a combination
# fun <- function to apply to the resulting output
# simplify <- if TRUE, print output as a matrix with m rows and nCm columns
  # else <- print output as a list 
  # nCm is the total number of combinations generated
# ... <- parameters for fun

# Helper functions for handling characters in input vector x:
# is.letter <- function to check if there's a char in x
# asc <- convert char to ASCII decimal value
# chr <- convert decimal value to ASCII character
# nCm <- calculate the total number of combinations 
  # taken directly from R combinat package nCm.R
  # inserted into this file saw combinat need not be installed when program is run
####################################################################################

combnthrust<- function(x, m, fun = NULL, simplify = TRUE, ...)
{
	require(Rcpp)
	#require(combinat) # necessary for nCm in line 24
	dyn.load("combnthrust.so")

	# Input checks taken directly from combn source code
	if(length(m) > 1) {
		warning(paste("Argument m has", length(m), 
			"elements: only the first used"))
		m <- m[1]
	}
	if(m < 0)
		stop("m < 0")
	if(m == 0)
		return(if(simplify) vector(mode(x), 0) else list())
	if(is.numeric(x) && length(x) == 1 && x > 0 && trunc(x) == x)
		x <- seq(x)
	n <- length(x)
	if(n < m)
		stop("n < m")

	nofun <- is.null(fun)
	count <- nCm(n, m, 0.10000000000000002)

	# Error checks for the scheduling variables: sched and chunksize
	# R handles the error when 'sched' is not a string/character vector
	
	# If sched is provided, then sched must be static, dynamic, guided, or NULL
	# if (!grepl('static', sched) && !grepl('dynamic', sched) && !grepl('guided', sched) && !is.null(sched)) {
	# 		stop("Scheduling policy must be static, dynamic, or guided.")
	# }
	# # Set to default values depending on what is/are provided
	# if (is.null(sched) && is.null(chunksize)) {
	# 	sched <- 'static'
	# 	chunksize <- 1
	# }
	# else if (!is.null(sched) && is.null(chunksize)) {
	# 	chunksize <- 1
	# }
	# else if (is.null(sched) && !is.null(chunksize)) { # if sched is provided, but chunk size is not
	# 	sched <- 'static'
	# 	warning("'sched' is replaced with default 'static' and 'chunksize' is overriden with default value.")
	# }

	# Checks if input vector x has characters
	# If so, then convert chars to their ASCII decimal values
	# Operate on the ASCII decimal values for the chars
	ischarx <- match('TRUE', is.letter(x))
	if (!is.na(ischarx)) {
		ischarx_arr <- is.letter(x)
		for (i in 1:length(ischarx_arr)) {
			if (ischarx_arr[i]) {
				if (length(asc(x[i])) == 1) {
					x[i] <- asc(x[i])
				}
				else {
					x[i] <- as.character(x[i])
				}
			}
			
		}
		x <- strtoi(x, base=10)
	}
	
	# Initialize output matrix
	retmat <- matrix(0, m, count)
	# Call the function through Rcpp
	retmat <- .Call("combnthrust", x, m, n, count)

	# Convert from ASCII decimal values back to chars if necessary
	if (!is.na(ischarx)) {
		for (i in 1:length(retmat)) {
			if ((as.integer(retmat[i]) >= 97 && as.integer(retmat[i]) <= 122)
			|| (as.integer(retmat[i]) >= 65 && as.integer(retmat[i]) <= 90)) {
				retmat[i] <- chr(retmat[i]);
			}
		}
	}

	# Apply provided function to the output
	if (!is.null(fun)) {
		apply(retmat, 2, fun(...))
	}

	# Format results
	if (simplify) {
		out <- retmat
	}
	else {
		out <- list()
		for (i in 1:ncol(retmat)) {
			out <- c(out, list(c(retmat[, i])))
		}
	}
	return(out)
}

# function to check if there's a char in x
is.letter <- function(x) grepl("[[:alpha:]]", x)
# convert char to ascii decimal value
asc <- function(x) { strtoi(charToRaw(x),16L) }
# convert decimal value to ascii character
chr <- function(n) { rawToChar(as.raw(n)) }

"nCm"<-
function(n, m, tol = 9.9999999999999984e-009)
{
#  DATE WRITTEN:  7 June 1995               LAST REVISED:  10 July 1995
#  AUTHOR:  Scott Chasalow
#
#  DESCRIPTION: 
#        Compute the binomial coefficient ("n choose m"),  where n is any 
#        real number and m is any integer.  Arguments n and m may be vectors;
#        they will be replicated as necessary to have the same length.
#
#        Argument tol controls rounding of results to integers.  If the
#        difference between a value and its nearest integer is less than tol,  
#        the value returned will be rounded to its nearest integer.  To turn
#        off rounding, use tol = 0.  Values of tol greater than the default
#        should be used only with great caution, unless you are certain only
#        integer values should be returned.
#
#  REFERENCE: 
#        Feller (1968) An Introduction to Probability Theory and Its 
#        Applications, Volume I, 3rd Edition, pp 50, 63.
#
	len <- max(length(n), length(m))
	out <- numeric(len)
	n <- rep(n, length = len)
	m <- rep(m, length = len)
	mint <- (trunc(m) == m)
	out[!mint] <- NA
	out[m == 0] <- 1	# out[mint & (m < 0 | (m > 0 & n == 0))] <-  0
	whichm <- (mint & m > 0)
	whichn <- (n < 0)
	which <- (whichm & whichn)
	if(any(which)) {
		nnow <- n[which]
		mnow <- m[which]
		out[which] <- ((-1)^mnow) * Recall(mnow - nnow - 1, mnow)
	}
	whichn <- (n > 0)
	nint <- (trunc(n) == n)
	which <- (whichm & whichn & !nint & n < m)
	if(any(which)) {
		nnow <- n[which]
		mnow <- m[which]
		foo <- function(j, nn, mm)
		{
			n <- nn[j]
			m <- mm[j]
			iseq <- seq(n - m + 1, n)
			negs <- sum(iseq < 0)
			((-1)^negs) * exp(sum(log(abs(iseq))) - lgamma(m + 1))
		}
		out[which] <- unlist(lapply(seq(along = nnow), foo, nn = nnow, 
			mm = mnow))
	}
	which <- (whichm & whichn & n >= m)
	nnow <- n[which]
	mnow <- m[which]
	out[which] <- exp(lgamma(nnow + 1) - lgamma(mnow + 1) - lgamma(nnow - 
		mnow + 1))
	nna <- !is.na(out)
	outnow <- out[nna]
	rout <- round(outnow)
	smalldif <- abs(rout - outnow) < tol
	outnow[smalldif] <- rout[smalldif]
	out[nna] <- outnow
	out
}
\end{lstlisting}
}


%%%%%%%%%%%%%%%%%%%%%%%%%%%%%%%%%%%%% THRUST C++ FILE %%%%%%%%%%%%%%%%%%%%%%%%%%%%%%%%%%%%%

{\lstset{
language=C++,
backgroundcolor = \color{white},
basicstyle = \scriptsize\ttfamily,
commentstyle=\color{mygreen},
stepnumber=1,
xleftmargin=2em,
framexleftmargin=1.8em,
numbers=left,
numbersep=5pt,
showspaces=false,
showstringspaces=false,
showtabs=false,
frame=single,
rulecolor=\color{black},
tabsize=2,
captionpos=b,
breaklines=true,
breakatwhitespace=false,
keywordstyle=\color{blue},
escapeinside={}
}


% THRUST C++ FILE

\begin{lstlisting}
/*********************************************************************************
Thrust (C++) implementation of R's combn() function from the CRAN combinat package

Called from combn-thrust.R using .Calll() through Rcpp interface
**********************************************************************************/
#include <thrust/host_vector.h>
#include <thrust/device_vector.h>
#include <thrust/random.h>
#include <stdlib.h>
#include <thrust/transform.h>
#include <stdio.h>
#include <boost/math/special_functions/binomial.hpp>

#include <Rcpp.h>

using namespace std; 
using namespace Rcpp;


struct comb {
	
	const thrust::device_vector<int>::iterator x;
	const thrust::device_vector<int>::iterator pos;
	const thrust::device_vector<int>::iterator retmat;
	int n;
	int m;
	int *x_arr, *position, *ret;


	comb(thrust::device_vector<int>::iterator _x_arr, thrust::device_vector<int>::iterator _pos, int _n, int _m, thrust::device_vector<int>::iterator _retmat):
		x(_x_arr),
		pos(_pos),
		n(_n),
		m(_m),
		retmat(_retmat)
	{
		x_arr = thrust::raw_pointer_cast(&x[0]);		
		position = thrust::raw_pointer_cast(&pos[0]);
		ret = thrust::raw_pointer_cast(&retmat[0]);
	}
	
	__device__
	void operator()(int i)
	{
        if(i <= n - m)
        {
            find_comb(i, x_arr, m, n, position, ret);
        }
	}
	__device__
	void store(int *pos, int *output, int idx, int m, int *x, int *comb, int &outidx){
		for(int i = 0; i < m; i++){
			output[outidx++] = x[comb[i]+idx];
		}

	}

	__device__
	void find_comb(int idx, int *x, int m, int n, int *pos, int *output){
		int *comb = new int[m];
		for(int i = 0; i < m; i++){
			comb[i] = i;
		}
		int new_n= n - idx;
        
        int outidx = pos[idx];
        
		store(pos, output, idx, m, x, comb, outidx);

		while(true){
			int i = m - 1;
			++comb[i];
	
			while((i >= 0) && (comb[i] >= new_n - m + 1 + i)){
				--i;
				++comb[i];
			}
			if(comb[0] == 1){
				break;
			}
			for(i = i + 1; i < m; ++i){
				comb[i] = comb[i-1] + 1;
			}
            store(pos, output, idx, m, x, comb, outidx);
		}
	}

};

RcppExport SEXP combnthrust(SEXP x_, SEXP m_, SEXP n_, SEXP nCm_, SEXP out){
	NumericVector x(x_);
	int m = as<int>(m_), n = as<int>(n_), nCm = as<int>(nCm_);
	NumericMatrix retmat(m, nCm);

    // keeps track of the position in output
    int *pos = new int[n-m+1];
    
    // Calculate combination possibilities for each element in the list that
    // start with the element in the 0th index
    int k = 0;
    for(int i = 0; i < (n-m+1); i++){
        pos[i] = boost::math::binomial_coefficient<double>(n - i - 1, m-1);
        k++;
    }
    // Calcluate the position vector respective to the possiblities
    int temp = pos[0]*m;
    int temp2;
    pos[0]=0;
    for (int j=1; j<(n-m+1); j++){
        temp2 = pos[j];
        pos[j]=temp;
        temp=pos[j]+(m*temp2);
    }
	
	thrust::device_vector<int> d_x(x.begin(), x.end());
    thrust::device_vector<int> d_pos(pos, pos + (n-m+1));
	thrust::device_vector<int> d_mat(retmat.begin(), retmat.end());
    
	thrust::counting_iterator<int> begin(0);
	thrust::counting_iterator<int> end = begin + n;

	thrust::for_each(begin, end, comb(d_x.begin(), d_pos.begin(), n, m, d_mat.begin()));

	thrust::copy(d_mat.begin(), d_mat.end(), retmat.begin());
	
	return retmat;
    
}


\end{lstlisting}
}

\newpage
% Account of work done %
\textbf{\Large{Account of Work Done}}\\
\null
Trisha Funtanilla: I worked on the OpenMP implementation, putting together the Rcpp interface, error checks, and output formatting for the codes, writing the OpenMP and Snow section in the report, and creating the timing comparison plots.
\\
\null
Syeda Inamdar: I worked with Jennifer on the Thrust implementation and integrating the Rcpp file Trisha wrote to the Thrust interface. I also worked with Jennifer to write the Thrust section in the report and conduct timing comparisons.
\\
\null
Eva Li: I worked on the OpenMP and Thrust implementations.
\\
\null
Jennifer Wong:



