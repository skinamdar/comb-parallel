\subsection{Snow}

\subsubsection{The Issue of Overhead}
After running tests using small inputs, it was observed that parallelization using Snow more often than not took longer than running the CRAN function \texttt{combn()} (sequentially). This is because of the communication overhead. The communication between the nodes in the cluster took more time than the actual computations in the function. If the jobs sent to the worker nodes are not relatively computationally extensive, the overhead of communicating ends up deteriorating the performance.

\subsubsection{Reducing Network Overhead}
In order to reduce network overhead and improve the performance, the code was written so that the nodes will do long calculations and less communications. The same load balancing algorithm as in Section 3.1.1 was used to distribute the tasks to each nodes.

\subsubsection{Comparative Analysis}
The same input sizes and function arguments as in Section 3.1.5 were used for testing. For the Snow tests, 8 clusters were used. The following plot illustrates the differences in speeds between the Snow and CRAN implementations.\\
\null

The following plots illustrate the difference in performance between the CRAN and Snow implementations:\\






