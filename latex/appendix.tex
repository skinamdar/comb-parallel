{\lstset{
language=R,
backgroundcolor = \color{white},
basicstyle = \scriptsize\ttfamily,
commentstyle=\color{mygreen},
stepnumber=1,
xleftmargin=2em,
framexleftmargin=1.8em,
numbers=left,
numbersep=5pt,
showspaces=false,
showstringspaces=false,
showtabs=false,
frame=single,
rulecolor=\color{black},
tabsize=2,
captionpos=b,
breaklines=true,
breakatwhitespace=false,
keywordstyle=\color{blue},
escapeinside={}
}

\begin{lstlisting}
#############################################################################
# R call function for the OpenMP parallelization of combn() from the CRAN
# combinat package: http://cran.r-project.org/web/packages/combinat/index.html

# NOTE: Output is out of order.

# Function Arguments:
# x <- input vector of integers and/or characters\
# m <- number of elements in a combination
# fun <- function to apply to the resulting output
# simplify <- if TRUE print output as a matrix with m rows and nCm columns
# where nCm is the total number of combinations generated
# ... <- parameters for fun

# Helper functions for handling characters in input vector x:
# is.letter <- # function to check if there's a char in x
# asc <- convert char to ASCII decimal value
# chr <- convert decimal value to ASCII character
# nCm <- calculating the total number of combinations 
	# (taken directly from R combinat package)
	# inserted into this file since combinat could not be installed in CSIF
#############################################################################

combn <- function(x, m, fun = NULL, simplify = TRUE, 
	sched = NULL, chunksize = NULL, ...)
{
	require(Rcpp)
	dyn.load("combn-final.so")

	# Input checks taken directly from combn source code
	if(length(m) > 1) {
		warning(paste("Argument m has", length(m), 
			"elements: only the first used"))
		m <- m[1]
	}
	if(m < 0)
		stop("m < 0")
	if(m == 0)
		return(if(simplify) vector(mode(x), 0) else list())
	if(is.numeric(x) && length(x) == 1 && x > 0 && trunc(x) == x)
		x <- seq(x)
	n <- length(x)
	if(n < m)
		stop("n < m")

	nofun <- is.null(fun)
	count <- nCm(n, m, 0.10000000000000002)

	# Error checks for the scheduling variables: sched and chunksize
	# R handles the error when 'sched' is not a string/character vector
	
	# If sched is provided, then sched must be static, dynamic, guided, or NULL
	if (!grepl('static', sched) && !grepl('dynamic', sched) && !grepl('guided', sched) && !is.null(sched)) {
			stop("Scheduling policy must be static, dynamic, or guided.")
	}
	# Set to default values depending on what is/are provided
	if (is.null(sched) && is.null(chunksize)) {
		sched <- 'static'
		chunksize <- 1
	}
	else if (!is.null(sched) && is.null(chunksize)) {
		chunksize <- 1
	}
	else if (is.null(sched) && !is.null(chunksize)) { # if sched is provided, but chunk size is not
		sched <- 'static'
		warning("'sched' is replaced with default 'static' and 'chunksize' is overriden with default value.")
	}

	# Checks if input vector x has characters
	# If so, then convert chars to their ASCII decimal values
	# Operate on the ASCII decimal values for the chars
	ischarx <- match('TRUE', is.letter(x))
	if (!is.na(ischarx)) {
		ischarx_arr <- is.letter(x)
		for (i in 1:length(charx)) {
			if (ischarx_arr[i]) {
				if (length(asc(x[i])) == 1) {
					x[i] <- asc(x[i])
				}
				else {
					x[i] <- as.character(x[i])
				}
			}
			
		}
		x <- strtoi(x, base=10)
	}


	#Calculate positions for output
	pos <- vector()
	temp_n <- n
	for (i in 1:(n-m+1)) {
		pos <- c(pos, nCm(temp_n-i, m-1))
	}
	temp <- pos[1]
	pos[1] <- 0
	for (i in 2:length(pos)) {
		temp2 <- pos[i]
		pos[i] <- temp
		temp <- pos[i] + temp2
	}

	# Initialize output matrix
	retmat <- matrix(0, m, count)
	# Call the function through Rcpp
	retmat <- .Call("combn", x, m, n, count, sched, chunksize, pos)

	# Convert from ASCII decimal values back to chars if necessary
	if (!is.na(ischarx)) {
		for (i in 1:length(retmat)) {
			if ((as.integer(retmat[i]) >= 97 && as.integer(retmat[i]) <= 122)
			|| (as.integer(retmat[i]) >= 65 && as.integer(retmat[i]) <= 90)) {
				retmat[i] <- chr(retmat[i]);
			}
		}
	}

	# Apply provided function to the output
	if (!is.null(fun)) {
		apply(retmat, 2, fun(...))
	}

	# Format results
	if (simplify) {
		out <- retmat
	}
	else {
		out <- list()
		for (i in 1:ncol(retmat)) {
			out <- c(out, list(c(retmat[, i])))
		}
	}
	return(out)
}

# function to check if there's a char in x
is.letter <- function(x) grepl("[[:alpha:]]", x)
# convert char to ascii decimal value
asc <- function(x) { strtoi(charToRaw(x),16L) }
# convert decimal value to ascii character
chr <- function(n) { rawToChar(as.raw(n)) }

# n choose m - calculates the total number of combinations for a given input
"nCm"<-
function(n, m, tol = 9.9999999999999984e-009)
{
#  DATE WRITTEN:  7 June 1995               LAST REVISED:  10 July 1995
#  AUTHOR:  Scott Chasalow
#
#  DESCRIPTION: 
#        Compute the binomial coefficient ("n choose m"),  where n is any 
#        real number and m is any integer.  Arguments n and m may be vectors;
#        they will be replicated as necessary to have the same length.
#
#        Argument tol controls rounding of results to integers.  If the
#        difference between a value and its nearest integer is less than tol,  
#        the value returned will be rounded to its nearest integer.  To turn
#        off rounding, use tol = 0.  Values of tol greater than the default
#        should be used only with great caution, unless you are certain only
#        integer values should be returned.
#
#  REFERENCE: 
#        Feller (1968) An Introduction to Probability Theory and Its 
#        Applications, Volume I, 3rd Edition, pp 50, 63.
#
	len <- max(length(n), length(m))
	out <- numeric(len)
	n <- rep(n, length = len)
	m <- rep(m, length = len)
	mint <- (trunc(m) == m)
	out[!mint] <- NA
	out[m == 0] <- 1	# out[mint & (m < 0 | (m > 0 & n == 0))] <-  0
	whichm <- (mint & m > 0)
	whichn <- (n < 0)
	which <- (whichm & whichn)
	if(any(which)) {
		nnow <- n[which]
		mnow <- m[which]
		out[which] <- ((-1)^mnow) * Recall(mnow - nnow - 1, mnow)
	}
	whichn <- (n > 0)
	nint <- (trunc(n) == n)
	which <- (whichm & whichn & !nint & n < m)
	if(any(which)) {
		nnow <- n[which]
		mnow <- m[which]
		foo <- function(j, nn, mm)
		{
			n <- nn[j]
			m <- mm[j]
			iseq <- seq(n - m + 1, n)
			negs <- sum(iseq < 0)
			((-1)^negs) * exp(sum(log(abs(iseq))) - lgamma(m + 1))
		}
		out[which] <- unlist(lapply(seq(along = nnow), foo, nn = nnow, 
			mm = mnow))
	}
	which <- (whichm & whichn & n >= m)
	nnow <- n[which]
	mnow <- m[which]
	out[which] <- exp(lgamma(nnow + 1) - lgamma(mnow + 1) - lgamma(nnow - 
		mnow + 1))
	nna <- !is.na(out)
	outnow <- out[nna]
	rout <- round(outnow)
	smalldif <- abs(rout - outnow) < tol
	outnow[smalldif] <- rout[smalldif]
	out[nna] <- outnow
	out
}
\end{lstlisting}
}

{\lstset{
language=C++,
backgroundcolor = \color{white},
basicstyle = \scriptsize\ttfamily,
commentstyle=\color{mygreen},
stepnumber=1,
xleftmargin=2em,
framexleftmargin=1.8em,
numbers=left,
numbersep=5pt,
showspaces=false,
showstringspaces=false,
showtabs=false,
frame=single,
rulecolor=\color{black},
tabsize=2,
captionpos=b,
breaklines=true,
breakatwhitespace=false,
keywordstyle=\color{blue},
escapeinside={}
}

\begin{lstlisting}
/****************************************************************************
OpenMP (C++) implementation of R's combn() function from the combinat package

Called from R using .Calll() through Rcpp
*****************************************************************************/

#include <Rcpp.h>
#include <omp.h>

using namespace std;
using namespace Rcpp;

// Computes the indices of the next combination to generate 
// The indices then get mapped to the actual values from the input vector
int next_comb(int *comb, int m, int n)
{
	int i = m - 1;	
	++comb[i];
		
	while ((i >= 0) && (comb[i] >= n - m + 1 + i)) {		
		--i;		
		++comb[i];	
	}
		
	if(comb[0] == 1) {
		return 0;
	}
		
	for (i = i + 1; i < m; ++i) {
		comb[i] = comb[i - 1] + 1;		
	}
	
	return 1;
}

RcppExport SEXP combn(SEXP x_, SEXP m_, SEXP n_, SEXP nCm_, SEXP sched_, SEXP chunksize_, SEXP pos_, SEXP out)
{
	// Convert SEXP variables to appropriate C++ types
	NumericVector x(x_); // input vector
	NumericVector pos(pos_); // position vector for the combinations so that the output is sorted
	int m = as<int>(m_), n = as<int>(n_), nCm = as<int>(nCm_), chunksize = as<int>(chunksize_);
	string sched = as<string>(sched_);

	NumericMatrix retmat(m, nCm);
	
	// OpenMP schedule clauses
	if (sched == "dynamic") {
		omp_set_schedule(omp_sched_dynamic, chunksize);
	}
	else if (sched == "guided") {
		omp_set_schedule(omp_sched_guided, chunksize);
	}

	if (sched == "static") { // use the load balancing algorithm
		#pragma omp parallel
		{
			// this thread id, total number of threads, combination indexes array
			int me, nth, *comb;

			nth = omp_get_num_threads();
			me = omp_get_thread_num();

			// array that will hold all of the possible combinations 
			// of size m of the indexes
			comb = new int[m]; 

			// initialize comb array
			for (int i = 0; i < m; ++i) {
				comb[i] = i;
			}
			
			int chunkNum = 1; // the number of chunk that has been distributed
			int mypos; // variable for the output position
			
			// each thread gets assign a chunk to work on
			// each thread will have about the same number of chunks
			// to work on throughout the lifetime of the program
			for(int current_x = me; current_x < n-m+1; current_x+=1) {
				int temp;
				mypos = pos[current_x];
				for (int i = 0; i < m; ++i) {
					temp = comb[i] + current_x;
					retmat(i, mypos) = x[temp];
				}	
				mypos++;
				while(next_comb(comb, m, n-current_x))  {
					int temp;
					for (int i = 0; i < m; ++i) {
						temp = comb[i] + current_x;
						retmat(i, mypos) = x[temp];
					}
					mypos++;
				}

				// reset comb array for the next chunk this thread will work on
				for(int i = 0; i < m; i++) {
					comb[i] = i;
				}

				chunkNum++; // increment chunkNum for the next chunk distribution
				// determine which element this thread will work on
				if (chunkNum % 2 == 0) {
					current_x = current_x + 2 * (nth - me - 1);
				}
				else {
					current_x = current_x + 2 * me;
				}
			}
		}
	}
	else { // dynamic or guided
		int mypos;
		#pragma omp parallel
		{
			int *comb = new int[m]; 
			for (int i = 0; i < m; ++i) {
				comb[i] = i;
			}
					
			#pragma omp for schedule(runtime)
			for(int current_x = 0; current_x < (n - m + 1); current_x++) {
				int temp;
				mypos = pos[current_x];
				for (int i = 0; i < m; ++i) {
					temp = comb[i] + current_x;
					retmat(i, mypos) = x[temp];
				}	
				mypos++;
				while(next_comb(comb, m, n-current_x))  {
					int temp;
					for (int i = 0; i < m; ++i) {
							temp = comb[i] + current_x;
							retmat(i, mypos) = x[temp];
					}	
					mypos++;
				}

				for(int i = 0; i < m; i++) {
					comb[i] = i;
				}
			}
		}
	}		
	return retmat;
}

\end{lstlisting}
}
